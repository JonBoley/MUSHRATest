%% LyX 2.0.2 created this file.  For more info, see http://www.lyx.org/.
%% Do not edit unless you really know what you are doing.
\documentclass[oneside,english]{book}
\usepackage[T1]{fontenc}
\usepackage[latin9]{inputenc}
\setcounter{secnumdepth}{3}
\setcounter{tocdepth}{3}
\usepackage{babel}
\begin{document}

\title{MUSHRATest}


\author{User's Guide}


\date{1 March 2012}

\maketitle

\section*{Introduction}

MUSHRATest is an implementation of the MUSHRA listening test defined
by ITU-R recommendation BS.1534-1%
\footnote{BS.1534-1. Method for the subjective assessment of intermediate quality
levels of coding systems.%
}. MUSHRATest was primarily intended for spatial audio psychoacoustic
tests and is thus optimised for multi-channel sound reproduction.

The software is written using JUCE library that supports a number
of target platforms including Mac OSX, iOS, Windows XP/Vista/Win7,
Linux and Android.


\section*{License}

The software is released under GNU General Public License Version
3.


\section*{Installation}

MUSHRTest can be downloaded from the author's website. The installers
are provided for 32- and 64 bit Windows targets. 

Choose the platform, start the installer and follow the instructions.
You should have administrators priveligies to install the program.


\section*{Configuration}

You will need to run MUSHRAConfig to setup the audio device and specify
subject's names and stimuli directories. It is recommended to use
ASIO audio interface (if supported by your hardware) and set the buffer
size to at least 512 samples.

MUSHRATest allows individualised stimuli to be used for each subject.
This was done to support binaural listening tests with the stimuli
rendered using individual subject's HRTFs. If this feature is not
required you can specify the same directory for each subject.


\section*{Preparing the stimuli}

There are certain rules and limitation in preparing your stimuli:
\begin{itemize}
\item All stimuli should be placed in one folder, with different subfolders
corresponding to each trial in the test. The name of the subfolder
will be displayed to the user during the tests.
\item Each stimulus in the trial should be saved as a (multichannel) WAV
file preferably in 32-bit floating point format%
\footnote{If you have your files in a different format you can use open-source
Audacity software to convert them. Visit \texttt{http://audacity.sourceforge.net/}
for more details. %
}. 
\item One of the files should be the reference and must have the word 'reference'
included in its filename. All other filenames could be arbitrary.
\end{itemize}

\section*{Running the tests}

MUSHRATest includes two parts: the training phase and evaluation phase.
The former is required by the standard and gives the users the opportunity
to familiarise themselves with the program interface and the stimuli
they are going to listen.


\section*{Results of the test}

By default, the results of the tests are saved in the same directory
as the subject's stimuli. If the stimuli directory does not provide
write access permissions than the file is saved in the User's Documents
directory.

The name of the results file has the following format: \texttt{{[}date{]}\_{[}time{]}\_{[}subject
name{]}.txt}. This is a tab-separated text file that can be read by
any text editor or spreadsheet software.


\section*{Compiling from the source code}

The source code and Visual Studio 2010 project files are available
from the author's website. The code depends on open-source JUCE library
that can be downloaded from \texttt{http://www.rawmaterialsoftware.com. }The
library should be placed in the same root folder as MUSHRATest. In
order to use ASIO audio device interface JUCE requires ASIO SDK to
be installed. This SDK can be obtained from \texttt{http://www.steinberg.net/en/company/developer.html}.
Depending on your computer configuration DirectX SDK might also be
required. This could also be obtained from Microsoft at: 

\texttt{http://msdn.microsoft.com/en-us/directx/aa937788}.


\section*{Bug reports and user feedback}

If you found a bug or have problems compiling the code please contact
the author using the form on the website. Please make sure that you
save the log file that the program creates. The log can be found in
the user-specific Application Data Directory. On Windows 7 64 bit
the full path is:

\texttt{C:\textbackslash{}Users\textbackslash{}<current-user>\textbackslash{}AppData\textbackslash{}Roaming\textbackslash{}MUSHRATest\textbackslash{}mushra.log.txt}
\end{document}
